\documentclass[letter]{scrartcl}
\usepackage{graphicx}
\usepackage{fullpage} %1in margins
\usepackage{tabularx}
\usepackage{hyperref}

\newcommand{\app}{\sc{393torrent}}

\begin{document}

\title{Requirements for \app}
\subtitle{Dan Keller, Kenneth Link, Nathan McKinley, Ross Nanopoulos}
\date{} % no date

\maketitle

\begin{abstract}
The requirements document specified all user stories (functional requirements) as well as nonfunctional requirements. These requirements are used in iteration planning and acceptance testing.
\end{abstract}

\tableofcontents
\pagebreak

\section{User Stories}
User stories are the functional requirements of XP. They are used in iteration planning as well as acceptance testing.
\\\\
\begin{tabularx}{\textwidth}{X c c c}
\textbf{User Story} & \href{http://c2.com/cgi/wiki?IdealProgrammingTime}{\textbf{Cost (Ideal Days)}}
& \textbf{Status} & \textbf{Iteration} \\
User wants to do \textit{things} and sometimes \textit{stuff}. & 4 days & Not Started & TBD \\
User wants to decide the halting problem. & \texttt{undefined} & Failed & \texttt{0x7F8FFFFF} \\
\end{tabularx}

\section{Nonfunctional Requirements}
Nonfunctional requirements are requirements that do not involve their own coding effort. They are used in acceptance testing, but not iteration planning. 
\subsection{Performance}
Don\rq{}t hog memory.
\subsection{Security}
Don\rq{}t let hackers into the user\rq{}s computer.
\subsection{Quality}
Don\rq{}t explode if something bad happens.

\pagebreak
\section{Author Contributions}
This section provides the details to what each author contributed to the vision and scope document, as requested per professor Podgurski.
\subsection{Daniel Keller}
Created document outline.
\subsection{Kenneth Link}
\subsection{Nathan McKinley}
\subsection{Ross Nanopoulos}
\end{document}
