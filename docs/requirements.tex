\documentclass[letter]{scrartcl}
\usepackage{graphicx}
\usepackage{fullpage} %1in margins
\usepackage{tabularx}
\usepackage{hyperref}

\newcommand{\app}{\sc{393torrent}}

\begin{document}

\title{Requirements for \app}
\subtitle{Dan Keller, Kenneth Link, Nathan McKinley, Ross Nanopoulos}
\date{} % no date

\maketitle

\begin{abstract}
The requirements document specified all user stories (functional requirements) as well as nonfunctional requirements. These requirements are used in iteration planning and acceptance testing.
\end{abstract}

\tableofcontents
\pagebreak

\section{User Stories}
User stories are the functional requirements of XP. They are used in iteration planning as well as acceptance testing.
\\\\
\begin{tabularx}{\textwidth}{X c c c}
\textbf{User Story} & \href{http://c2.com/cgi/wiki?IdealProgrammingTime}{\textbf{Cost (Ideal Days)}}
& \textbf{Status} & \textbf{Iteration} \\
Open .torrent file and initiate download \\
Communicate with tracker \\
Communicate with peers \\
Track peers \\
Track and assemble file pieces \\
Test piece checksums \\
Calculate percent complete \\
Calcuate download speed \\
Calcuate upload speed \\
Estimate ETA \\
Display progress bar \\
Interpret peer piece request \\
Send file pieces to peer \\
Saving state of download to file \\
Resuming download from saved state \\
Continuously save state in case process is killed \\
Accept a list of files and output a .torrent \\
Track or store local files being uploaded \\
Ability to download multiple files via multiple instances \\
\href{https://wiki.theory.org/BitTorrentSpecification\#have:_.3Clen.3D0005.3E.3Cid.3D4.3E.3Cpiece_index.3E}{HAVE suppression} \\
\href{https://wiki.theory.org/BitTorrentSpecification\#Queuing}{Queueing} \\
\href{https://wiki.theory.org/BitTorrentSpecification\#Super_Seeding}{Super Seeding} \\
\href{https://wiki.theory.org/BitTorrentSpecification\#Piece_downloading_strategy}{Piece Download Strategy} \\
\href{https://wiki.theory.org/BitTorrentSpecification\#End_Game}{End Game} \\
\href{https://wiki.theory.org/BitTorrentSpecification\#Choking_and_Optimistic_Unchoking}{Choking} \\
Fast Peer extension \\
DHT extension \\
Encryption extension \\
Magnet links \\
Human-readable config file
\end{tabularx}

\section{Nonfunctional Requirements}
Nonfunctional requirements are requirements that do not involve their own coding effort. They are used in acceptance testing, but not iteration planning. 
\subsection{Performance}
Don\rq{}t hog memory.
\subsection{Security}
Don\rq{}t let hackers into the user\rq{}s computer.
\subsection{Quality}
Don\rq{}t explode if something bad happens.

\pagebreak
\section{Author Contributions}
This section provides the details to what each author contributed to the vision and scope document, as requested per professor Podgurski.
\subsection{Daniel Keller}
Created document outline.
\subsection{Kenneth Link}
\subsection{Nathan McKinley}
\subsection{Ross Nanopoulos}

\section{Inspection}
\begin{tabularx}{\textwidth}{X c c}
\textbf{Comment} & \textbf{Reported By} & \textbf{Fixed By} \\
\end{tabularx}
\end{document}
