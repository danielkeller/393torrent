\documentclass[letter]{scrartcl}
\usepackage{graphicx}
\usepackage{fullpage} %1in margins
\usepackage{tabularx}

\newcommand{\app}{\sc{393torrent}}

\begin{document}

\title{Vision and Scope Document}
\subtitle{for \app}
\date{} % no date

\maketitle

\begin{abstract}
The Vision and Scope document outlines high level vision and scope of our software project.  The document describes primary purpose and functionality of the application, as well as objectives and timeline for the project. 
\end{abstract}

\tableofcontents
\pagebreak

\section{Business Requirements}
\subsection{Background, Business Opportunity, and Customer Needs}
Many heavyweight BitTorrent clients currently exist that are littered with ads, masking the primary functionality for the user.

\subsection{Business Objectives and Success Criteria}
%this creates a labeled list
\begin{description}
\item[BO-1] Follow BitTorrent specification
\item[BO-2] Create usable alternative to mainstream BitTorrent clients
\item[BO-3] Perform the most important functions of BitTorrent clients
\item[BO-4] Needs to be fast:
Load in \textless500ms,
Respond to user in \textless100ms.
\item[SC-1] Group members can use the program for their day-to-day BitTorrent tasks
\end{description}

\subsection{Business Risks}
\begin{description}
\item[RI-1] Software bugs may potentially compromise security
\item[RI-2] Software may crash during computation of simultaneous downloads
\end{description}


\section{Vision of the Solution}
\subsection{Vision Statement}
The project's purpose is to create a lightweight BitTorrent application free of advertisements, providing ease of use and ultimate productivity for the user.  Users will be able to upload and download files, create torrents, and configure the client without the inconvenience of advertisements.
\subsection{Major Features}
%numbered list
\begin{enumerate}
\item Ability to download a file
\item Ability to display progress to user
\item Ability to upload a file
\item Ability to pause/resume, including across program runs
\item Ability to create torrents
\item Ability to download multiple torrents at once
\item Optimal download speed with use of appropriate algorithms
\item Official extensions - fast peer, dht, encryption
\item Support for magnet links
\item Persistent configuration
\end{enumerate}

\subsection{Assumptions and Dependencies}
\begin{description}
\item[AS-1] Specification documentation is correct
\item[AS-2] Common existing clients are compatible with the BitTorrent specification and thus intercompatibility will be possible.
\item[DE-1] Client must be compatible with Windows, Linux, iOS;
System will be developed as a command line program; time permitting, a GUI may be added.
\end{description}

\section{Scope and Limitations}
\subsection{Scope of Initial and Subsequent Releases}
%this is how you do a table. the ells are left alignment and the bars are vertical lines
\begin{tabularx}{\textwidth}{| X | X | X | X | X | X | X |}
\hline
\textbf{Feature} & \textbf{V1} & \textbf{V2} & \textbf{V3} & \textbf{V4} & \textbf{V5} \\
\hline
\hline
1 & Done & & & & \\
\hline
2 & Partly Done & Done & & & \\
\hline
3 & & Done & & & \\
\hline
4 & & & Done & & \\
\hline
5 & & & & Done & \\
\hline
6 & & & & & Done \\
\hline
7 & & & Done & & \\
\hline
8 & & & & Done & \\
\hline
9 & & Done & & & \\
\hline
10 & & & & & Done \\
\hline
\end{tabularx}

%\subsection{Limitations and Exclusions}
%i`m pretty sure this isn`t applicable


\section{Business Context}
\subsection{Stakeholder Profiles}
\tabcolsep=0.11cm
\begin{tabularx}{\textwidth}{| X | X | X | X | X |}
\hline
\textbf{Stakeholder} & \textbf{Major Value} & \textbf{Attitudes} & \textbf{Major Interests} & \textbf{Constraints} \\
\hline
\hline
Developers & Creation of the BitTorrent client & Commitment through all versions & Quality, performance, maintainability, scalability & Most members learning Haskell for the first time \\
\hline
Seeders & Give downloaders opportunity to download files; more seeders allow faster download times & Allow others to download file from them & Providing service to downloaders &  \\ 
\hline
Downloaders & People who download files from seeders & May or may not become seeders or leechers & Getting a specific file for their use & Number of seeders \\
\hline
Leechers& None, they do not contribute anything & Benefit from downloaded files, but do not offer to seed & Acquiring information and files & \\
\hline
\end{tabularx}

\subsection{Project Priorities}
\begin{tabularx}{\textwidth}{| X | X | X | X |}
\hline
\textbf{Dimension} & \textbf{Driver} & \textbf{Constraint} & \textbf{Degree of Freedom} \\
\hline
\hline
Schedule & & Limited to one semester of work; must hit project milestones for given dates & A new release must be delivered every week \\
\hline
Features & Getting features right means users are satisfied & In each release, features for that release must be fully operational & Features for specified release must be included and work \\
\hline
Quality  & & All unit tests must pass; all security tests must pass; compliance with software specification must be enforced & Test-driven development will be utilized to ensure quality\\
\hline
Scalability & Code will be able to add features and extensions as needed & For the client to be lightweight, scope must be limited & \\
\hline
Maintainability & Code will follow good programming practices & Maintainable code depends largely on the discipline of the developers &  \\
\hline
Performance & Code will maintain the performance specifications as stated in the business objectives &  & \\
\hline
Scalability & & Can download payloads of any size without using over 50MB of RAM & \\
\hline
Members  & 4 developers who will all write code and test & Many new to functional programming & \\
\hline
\end{tabularx}
\pagebreak
\section{Author Contributions}
This section provides the details to what each author contributed to the vision and scope document, as requested per professor Podgurski.
\subsection{Daniel Keller}
Daniel took care of the LaTeX template, business objectives, success criteria, major features list, and scope of initial and subsequent releases. He also fixed LaTeX errors.
\subsection{Kenneth Link}
Kenneth provided intellectual insight during brainstorming of the document needs and completed proofreading and editing of the final document.
\subsection{Nathan McKinley}
Nathan added additional content to Assumptions and Features, and made small edits for wording.
\subsection{Ross Nanopoulos}
Ross took care of writing the vision statement, stakeholder profiles, and the project priorities.  He also adjusted the phrasing of features and business objectives and added to business risks.
\end{document}
