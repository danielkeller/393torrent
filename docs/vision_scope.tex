\documentclass[letter]{scrartcl}
\usepackage{graphicx}
\usepackage{fullpage} %1in margins

\newcommand{\app}{\sc{393torrent}}

\begin{document}

\title{Vision and Scope Document}
\subtitle{for \app}
\date{} % no date

\maketitle

\begin{abstract}
This document outlines the high level plan for our software project.
\end{abstract}

\tableofcontents
\pagebreak

\section{Business Requirements}
\subsection{Background, Business Opportunity, and Customer Needs}
Many heavyweight BitTorrent clients currently exist that are littered with ads, masking the primary functionality for the user.

\subsection{Business Objectives and Success Criteria}
%this creates a labeled list
\begin{description}
\item[BO-1] Follow BitTorrent specification
\item[BO-1] Create usable alternative to mainstream BitTorrent clients
\item[BO-2] Perform the most important functions of BitTorrent clients
\item[BO-3] Needs to be fast 
\begin{enumerate}
  \item Load in \textless500ms
  \item Respond to user in \textless100ms
\end{enumerate}
\\
\item[SC-1] Group members can use the program for their day-to-day BitTorrent tasks
\end{description}

\subsection{Business Risks}
\begin{description}
\item[RI-1] Software bugs could potentially compromise security
\end{description}


\section{Vision of the Solution}
\subsection{Vision Statement}
To create a lightweight BitTorrent application free of ads, providing ease of use and ultimate productivity for the user.
\subsection{Major Features}
%numbered list
\begin{enumerate}
\item Ability to download a file
\item Ability to upload a file
\item Ability to submit user feedback
\item Ability to pause/resume
\item Ability to create torrents
\item Ability to download multiple torrents at once
\item Optimal download speed with use of appropriate algorithms
\item Official extensions - fast peer, dht, encryption
\item Magnet links
\item Configurable
\end{enumerate}

\subsection{Assumptions and Dependencies}
\begin{description}
\item[AS-1] Specification documentation is correct
\end{description}

\section{Scope and Limitations}
\subsection{Scope of Initial and Subsequent Releases}
%this is how you do a table. the ells are left alignment and the bars are vertical lines
\begin{tabular}{| l | l | l | l | l |}
\hline
\textbf{Feature} & \textbf{Version 1} & \textbf{Version 2} & \textbf{Version 3} & \textbf{Version 4} \\
\hline
\hline
1 & Fully implemented & & & \\
\hline
2 & etc & & & \\
\hline
\end{tabular}

%\subsection{Limitations and Exclusions}
%i`m pretty sure this isn`t applicable


\section{Business Context}
\subsection{Stakeholder Profiles}
seeders, dowloaders

\subsection{Project Priorities}
\begin{tabular}{| l | l | l | l | l |}
\hline
\textbf{Dimension} & \textbf{Driver} & \textbf{Constraint} & \textbf{Degree of Freedom} \\
\hline
\hline
Schedule & & & \\
\hline
Features & & & \\
\hline
Quality  & & & \\
\hline
Members  & & & \\
\end{tabular}
\end{document}
